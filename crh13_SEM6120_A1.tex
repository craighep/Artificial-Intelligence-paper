
\documentclass[article]{IEEEtran}
\usepackage{graphicx}


\begin{document}

\title{Fundamentals of Artificial Intelligence Assignment 1- Ethical issues of AI}

\author{\IEEEauthorblockN{Craig Heptinstall}
\IEEEauthorblockA{Crh13- 110005643\\SEM6120\\
Institute of Computer Science\\
Aberystywth University}}

\maketitle

\begin{abstract}
The ethics of Artificial intelligence is an ever increasing debate amongst scientists and those that try to enforce new laws (e.g. privacy) or rights, with machines becoming more and more involved with the day-to-day practises and activities of people around the world. Though true AI (there is debate about that too) is something that may take many years to conceive, the ethics will always be an issue that will need addressing. So what is AI? Why are ethics important? What ethics are already considered today?
\end{abstract}

\section{Introduction}
Although this paper looks generally at ethics in Artificial intelligence, to understand the topic in more depth, ethics can be broken into two questions that can be asked for both humans and machines alike. Can the subject (person or machine) be ethical? And can the subject be effected by the ethics of others? The first of these questions looks at the responsibilities of AI, and whether it should be used for such purposes as weaponry, or health care. The second question looks at AI rights, laws that could one day effect how they work, and if AI can even be considered to be under the same social circumstances as a person.

\subsection{What is Artificial Intelligence?}
Some text.

\subsection{The Ethics of AI}
Some Text.

\subsection{Why Ethics are important}
Some Text

\section{Recent research into Ethics of AI}
Stuff here, break it up.

\section{A critique of Ethics research}
Break this up.

\section{Overall findings and conclusion}

\begin{thebibliography}{1}

\bibitem{IEEEhowto:kopka}
H.~Kopka and P.~W. Daly, \emph{A Guide to \LaTeX}, 3rd~ed.\hskip 1em plus
  0.5em minus 0.4em\relax Harlow, England: Addison-Wesley, 1999.

\end{thebibliography}
\end{document}



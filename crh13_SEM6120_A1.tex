
\documentclass[article]{IEEEtran}
\usepackage{graphicx}


\begin{document}

\title{Fundamentals of Artificial Intelligence Assignment 1- Ethical issues of AI}

\author{\IEEEauthorblockN{Craig Heptinstall}
\IEEEauthorblockA{Crh13- 110005643\\SEM6120\\
Institute of Computer Science\\
Aberystywth University}}

\maketitle

\begin{abstract}
The ethics of Artificial intelligence is an ever increasing debate amongst scientists and those that try to enforce new laws (e.g. privacy) or rights, with machines becoming more and more involved with the day-to-day practises and activities of people around the world. Though true AI (there is debate about that too) is something that may take many years to conceive, the ethics will always be an issue that will need addressing. So what is AI? Why are ethics important? What ethics are already considered today?
\end{abstract}

\section{Introduction}
Although this paper looks generally at ethics in Artificial intelligence, to understand the topic in more depth, ethics can be broken into two questions that can be asked for both humans and machines alike. Can the subject (person or machine) be ethical? And can the subject be effected by the ethics of others? The first of these questions looks at the responsibilities of AI, and whether it should be used for such purposes as weaponry, or health care. The second question looks at AI rights, laws that could one day effect how they work, and if AI can even be considered to be under the same social circumstances as a person.

\subsection{What is Artificial Intelligence?}
Before answering questions about the ethics of AI, it is important to define to a certain extent what the term ‘Artificial intelligence’ means. To break this further, understanding the word intelligence can help define the greater meaning. Alan Turing \cite{Alan Turing website} breaks intelligence into five major components, all of which should be fulfilled in order to be classed as intelligent:
\begin{itemize}
\item Learning- The simplest form of this is trial and error, with more complicated forms such as generalisation meaning the learner can perform better in situations not encountered before.
\item Reasoning- Using evidence from a set of given statements to deduct a conclusion.
\item Problem solving- Special and general-purpose methods exists, where the special means a method of solving the problem is tailor made, while the latter means the method can be applied to a larger pool of general problems.	
\item Perception- To be able to process and analyse scenes into objects, features and relationships.
\item Language- To use a system of signs, or sounds to communicate or send information to others. 
\end{itemize}
Knowing the general requirements of intelligence, Artificial Intelligence should allow machines perform operations or actions that require the intelligence listed above in humans.


\subsection{The Ethics of AI}
Some Text.

\subsection{Why Ethics are important}
Some Text

\section{Recent research into Ethics of AI}
Stuff here, break it up.

\section{A critique of Ethics research}
Break this up.

\section{Overall findings and conclusion}

\begin{thebibliography}{9}

\bibitem{Alan Turing website} 
What is AI?,
\\
%\texttt{www.alanturing.net/turing_archive/pages/reference articles/what is ai.htmll}
\end{thebibliography}
\end{document}


